\documentclass{report}

\usepackage[utf8]{inputenc}
\usepackage{gensymb}
\usepackage{hyperref}
\usepackage{etoolbox}
\usepackage{graphicx}
\usepackage{geometry}
\edef\restoreparindent{\parindent=\the\parindent\relax}
\usepackage[parfill]{parskip}
\restoreparindent
\usepackage{indentfirst}

\title{Developer's Guide}
\author{BERQUEZ Fabien  (1800325), LOGUT Adrien (1815142) \\ \& NAMALGAMUWA Chathura (1815118)}
\date{}

\begin{document}
\maketitle

\tableofcontents

\newtoggle{diag}

\toggletrue{diag}
%\togglefalse{paper}

\chapter{Introduction}

Hi there ! Welcome to the developer's guide of \textbf{arch-LOG8430} - yes we are fully aware that this name sucks-. \\

This project was developed for the \href{http://www.polymtl.ca/etudes/cours/details.php?sigle=LOG8430}{Software Architecture} course (LOG8430) at Ecole Polytechnique de Montréal. The course teacher is Yann-Gaël Guéhéneuc and the laboratory teacher is Fábio Petrillo. \\

We are three French exchange students; Fabien Berquez coming from \href{http://www.supelec.fr/374_p_14603/welcome.html}{Supélec} and Adrien Logut and Chathura Namalgamuwa coming from \href{http://ensimag.grenoble-inp.fr/welcome/}{ENSIMAG}. Fabien is a master's student in computer engineering, research-based, whereas Adrien and Chathura are master's students in computer engineering, professional-based. \\ 

\ \\

Note : We advise you to use a \textit{real} pdf reader (and not the one integrated to your web browser for instance) as there are a lot of clickable links that might not be displayed otherwise.
\chapter{Presentation}

You can find all complete assignments (in French) in the \textit{assignment} folder. The objective of this project is to create a multi-platform web music player (eg. Soundcloud, Spotify, etc.) in Java. The source code of this project and all it's documentation are available on \href{https://github.com/cnamal/arch-LOG8430}{GitHub}.

\section{Laboratory n$\degree$1}

The first laboratory's objective is to create a basis for the whole project. A basic music player has to be implemented, using at least 3 web services. \\

We started with implementing a service for Soundcloud because it was the easiest. Indeed, Soundcloud's Web API can be used without authentication. It allowed us to test the code really quickly just using a \textit{client\_id}. 

%\section{Laboratory n$\degree$2}
%\section{Laboratory n$\degree$3}

\section{Requirements}

\begin{enumerate}
\item Java 8
\item Git
\item Mac/Linux : maven
\item (optional) e(fx)clipse for Eclipse
\item (optional) ObjectAid for Eclipse
\end{enumerate}

We decide to use maven because early on we knew we were going to use external libraries -especially \href{http://www.javazoom.net/javalayer/javalayer.html}{JLayer}- and using maven is \textit{way} simpler.

\section{Usage}


\subsection{Linux}

\begin{enumerate}
\item Clone the \href{https://github.com/cnamal/arch-LOG8430}{repository}
\item \textit{cd} in the project folder
\item Run \textbf{mvn install}
\item If you use Eclipse, also run \textbf{mvn eclipse:eclipse}
\item Profit!
\end{enumerate}

\subsection{Windows}

\subsection{Eclipse}
Normally, you shouldn't have any problems using Eclipse if you have installed e(fx)clipse (and done \textbf{mvn eclipse:eclipse} on Mac/Linux). However, if you have any problems, make sure the project is configured to use Java 8 (maven should do that for you).

\subsection{ObjectAid}
ObjectAid is an Eclipse plugin that automatically draws the UML diagrams of your project. If you have installed it, you can use it to open the \textbf{*.ucls} files in the documentation folder.
\chapter{Sequence Diagram}

\section{Laboratory n$\degree$1}

The figures \ref{play} and \ref{search} are  play sequence and a search sequence respectively.
\iftoggle{diag}{
\begin{figure}[H]

  \includegraphics[scale=0.4]{seq/PlaySequence.pdf}
  \caption{Play sequence}
  \label{play}
\end{figure}

\begin{figure}[H]

  \includegraphics[scale=0.4]{seq/SearchSequence.pdf}
  \caption{Search sequence}
  \label{search}
\end{figure}
}

%\section{Laboratory n$\degree$2}
%\section{Laboratory n$\degree$3}

\chapter{Class Diagram}

\section{Laboratory n$\degree$1}
\subsection{Methodology}

When we chose to do this option (two project options were given), we started to think about the \href{https://s3.amazonaws.com/applause-devmktg/2015/12/02/57kz3goral_YOU_DONT_SAY.png}{architecture of the project}. We started to draw a few UML diagrams on a piece of paper highlighting the main models and a few pattern designs we would use. At the end of that mini work-session, we knew we would use a MVC architecture, and as models we would have \textit{Song} and \textit{Playlist}, as controllers we'd have the \textit{Player} and finally we would have \textit{AudioService(s)} that would be generic. To be able to have numerous audio services, the idea to use the \href{https://en.wikipedia.org/wiki/Strategy_pattern}{\textit{strategy pattern}} was most obvious. However, after this first session, we never drew a UML diagram again. Indeed, the rest of the architecture was designed directly by coding the classes (cf \href{https://github.com/cnamal/arch-LOG8430/commit/d4953e0fc8fc3a7a8641dd7cacef69496f3ca011}{initial commit}). By coding the classes, we mean that every class was created along with it's methods and attributes, but the methods were not implemented. A few \texttt{TODO}s were added to explain what a method should do, and hence explain the main workflow.\\

This way of working has had a lot of advantages but also a few drawbacks (from least to most pertinent): 
\begin{enumerate}
\item \textbf{Advantages}
\begin{itemize}
\item Less boring. Even though this might appear as a joke, it isn't \tiny{(okay maybe a bit)}.\normalsize  However, directly coding gives better sense of progress. The \textit{boring} task of drawing UMLs is replaced by a more fun activity with more or less the same effect. 
\item More productivity. The fact that the exercise is now "fun" makes you want to code more and thus gain in productivity.
\item Kill two birds with one stone. If we had drawn UML diagrams we would have needed to \textit{transcribe} it in Java classes. Granted, it isn't the longest task.
\item Somehow easier. Directly coding allows to find a few bugs or conception flaws we think we wouldn't have had if we had drawn it. Indeed, you are immersed in the code instead of having a layer of abstraction, that are UML classes. One could however argue that the goal was to learn how to immerse yourself using UMLs. It would be an interesting debate.
\end{itemize}
\item \textbf{Disadvantages}
\begin{itemize}
\item Extracting class diagrams. For this guide, we have to add a few diagram classes and recreating UML diagrams feel more of a step back. However, as good lazy developers, we found an Eclipse plugin that did that for us.
\item We are less prone to creating the \textit{perfect} architecture. Being too immersed in the code gives us even more the desire to start coding. Therefore, we could miss a few conception flaws. This happened with the \texttt{Soundcloud} class (cf \hyperlink{blob}{blob issue})
\item Teamwork is harder. The problem is only one person did the whole architecture and \textbf{then} it was debated with the other team members. Therefore, one person \textit{forces} his point of view on the others. Even though the other have the possibility to object, it requires a greater effort of thinking outside the box.
\end{itemize}

Overall, we feel like the advantages outweighed the disadvantages, and therefore we stick by this choice. It allowed us to have a functional player really quickly.

\end{enumerate}

\subsection{Figures}

The figures \ref{Controller} and \ref{Services} are class diagrams of the Controller package and Services package respectively.

\iftoggle{diag}{
\begin{figure}[H]

  \includegraphics[scale=0.4]{class/Controller.png}
  \caption{Controller package diagram}
  \label{Controller}
\end{figure}

\begin{figure}[H]

  \includegraphics[scale=0.4]{class/Services.png}
  \caption{Services package diagram}
  \label{Services}
\end{figure}
}

\subsection{Explanations}

\begin{itemize}
\item The MVC architecture allowed us to have a modular project, and everyone worked on a part : Adrien on the view, Fabien on the controller and Chathura on the models/services. 
\item The models have \textit{pure} models and services. This choice is purely aesthetical. The services correspond to the Web services whereas the models correspond to really abstract objects. We could have created only the model package.
\item The \href{https://en.wikipedia.org/wiki/Strategy_pattern}{strategy pattern} allows us to offer an interface for any Audio service. Additionally, one can notice the fact that this pattern allows us to use any type of audio service, whether it is Web-based or your local machine.
\item The \href{https://en.wikipedia.org/wiki/Observer_pattern}{observer pattern} is used as a callback mechanism especially between the view and the controller (\texttt{PlayerController}), but also within the controller (between \texttt{PlayerController} and \texttt{PausablePlayer}). The observer in the \texttt{PlayerController} uses interfaces, since \textit{any} class could want to observe the player. Furthermore, the subject is the \texttt{PlayerController} and not the \texttt{PausablePlayer} because the latter is destroyed/created for every song. We haven't use any interfaces between \texttt{PlayerController} and \texttt{PausablePlayer} since only the \texttt{PlayerController} observes the \texttt{PausablePlayer}.
\item \href{http://weknowmemes.com/generator/uploads/generated/g1406353714448670979.jpg}{Singletons} have been used profusely. It allows us to have only one \textit{entry point} -like the Soundcloud class-, and it solves a lot of potential errors : for example having multiple players could lead to multiple songs being played at the same time, or worse multi-threading issues which would have been very hard to debug. 
\end{itemize}

%\section{Laboratory n$\degree$2}
%\section{Laboratory n$\degree$3}
\chapter{Implementation}

\section{Laboratory n$\degree$1}

\subsection{View}
In the MVC model, the view (ie. Graphical User Interface or GUI) is implemented apart the others. GUI was implemented with the pure API Java called JavaFX. 
Everything was implemented in the package view (for Java classes) and the JavaFX resources such as FXML files, CSS files and images are in the folder resources. 

We had the choice between using an GUI (called SceneBuilder) to create our own GUI and save it into FXML files or to code it from scratch in a Java Controller. At the end, both methods were used. In order to organize our view, we used SceneBuilder. And when the view was "repetitive" it was done with Java.

We need for each of or view a controller in order to control our view and react to the user inputs and actions.

\subsubsection{Main UI Class}
The main UI class is essential, everything is loaded from it and each controller has a reference to this class, to interact with other components of the UI. It extends the Application class, used by JavaFX.

\subsubsection{UI Controller}
Each controller of the UI extends the UI Controller which is a simple class with a reference to the UI element(Pane class for JavaFX), and 2 methods used to load a FXML file. Since almost every controller need the FXML loader, it seems it was useful and cleaner to refactor it a parent class.

\subsubsection{Overview Controllers}
The idea here was to create little modules of each function we want to implement instead of a big controller with everything in it. For example, the General Layout, which is the first thing we see is divided in 3 parts. The list at the left, controlled by the controller associated to General Layout, the part top right where the modules from the list will be loaded and the bottom where the player is. Each component is independent and can be loaded where we want. It allows a better maintainability and the possibility to use one module elsewhere without changing anything.

\subsection{Controller}

\subsubsection{Player}

The player we implemented uses the \href{http://www.javazoom.net/javalayer/javalayer.html}{JLayer} library. However, this player had limitations, notably the fact that we couldn't pause the player. We therefore adapted a code found on \href{http://stackoverflow.com/questions/12057214/jlayer-pause-and-resume-song}{Stackoverflow}. It already had a pausing function and the thread handling was already done. The observers and a stopping function were added.

\subsection{Models}

\subsubsection{Bob the builder}
Songs have a lot attributes and worse, some are optional! A solution could have been to have a default constructor with all the mandatory attributes, and setters for all the optional parameters. However, there were still \textit{too many} compulsory attributes. This \href{http://stackoverflow.com/a/40324/5795409}{Stackoverflow answer} offers a neat solution -the articles it refers to are great-. We ended up doing a mix between the source code suggested in the answer and a \href{https://dzone.com/articles/factories-builders-and-fluent-}{\texttt{Fluent interface pattern}}.

\subsubsection{Strategy Pattern}
The main interface is \texttt{AudioService},  and \texttt{AudioServiceProvider} and \texttt{IAuthentification} are just helper interfaces to have a distinct interfaces for distinct parts. It offers a certain modularity. \\
The \texttt{AudioServiceLoader} is a singleton class that acts as a bridge between the view and the services. 

%\section{Laboratory n$\degree$2}
%\section{Laboratory n$\degree$3}
\chapter{Limitations}

\section{Laboratory n$\degree$1}
\begin{itemize}
\item The authentication services were rushed towards the end, so they are not perfect. For instance, a failed authentication doesn't throw an exception but only returns false, or if the Soundcloud account is modified (especially if the user's password is modified), the program will not handle it -but shouldn't crash-. The only solution is to \href{http://img.pandawhale.com/post-16780-have-you-tried-forcing-an-unex-uQSY.gif}{disconnect and reconnect}. 
\item Even though they weren't yet implemented, Spotify and Deezer services don't allow third-party apps to stream their music, even if the user is authenticated. Only 30 seconds of the song will be played. Playlist management and song displays should however be functional.
\item Threads are only created for the player. Therefore, any communication with a Web service is blocks the main thread.
\item Testing is very hard for this project, hence the lack of tests. Indeed, a lot of testing requires user interaction (even without the view).
\item \hypertarget{db}{The architecture}, \textit{it's present form}, doesn't work if we want to save the data in a database. For instance, the Song model has a reference to a \texttt{AudioServiceProvider}. This flaw was discovered talking to another group who work on the same project as us. For now, the architecture is good enough for the objectives we fixed ourselves.Furthermore, we don't think the tweaks needed should be too difficult to add.
\end{itemize}

%\section{Laboratory n$\degree$2}
%\section{Laboratory n$\degree$3}
\chapter{Future fixes}

\section{Laboratory n$\degree$1}
\begin{enumerate}
\item Urgent 
\begin{itemize}
\item \hypertarget{blob}{Souncloud} has become a \textit{blob} class. We should be able to break the class in different \textit{sub-classes} since the interfaces are already done. The possible issue is if the sub-classes are dependant on each other.
\item The authorization process is a bit messy and is currently entirely implemented in the Soundcloud class. However, the authorization process is the same for Souncloud and Spotify (and possibly Deezer), so the code can be refactored in a helper \textit{generic} class.
\end{itemize}
\item Intermediate
\begin{itemize}
\item Create cross-platform playlists. Beside the obvious difficulty of the task, the current architecture might need to be slightly \hyperlink{db}{modified}.
\end{itemize}
\item Hopefully
\begin{itemize}
\item Create threads for web queries. This is very complicated and isn't necessarily worth it. Indeed, if you have a good internet connection, the waiting times aren't bad. However, once we will add other services, this might become necessary.
\end{itemize}
\end{enumerate}

%\section{Laboratory n$\degree$2}
%\section{Laboratory n$\degree$3}
\end{document}