\documentclass{report}

\usepackage[utf8]{inputenc}
\usepackage{gensymb}
\usepackage{hyperref}

\title{Developer's Guide\\ Nama}
\author{BERQUEZ Fabien  (1800325), LOGUT Adrien (1815142) \\ \& NAMALGAMUWA Chathura (1815118)}
\date{}

\begin{document}
\maketitle

\tableofcontents

\chapter{Introduction}

Hi there ! Welcome to the developer's guide of \textbf{arch-LOG8430} - yes we are fully aware that this name sucks-. \\

This project was developed for the \href{http://www.polymtl.ca/etudes/cours/details.php?sigle=LOG8430}{Software Architecture} course (LOG8430) at Ecole Polytechnique de Montréal. The course teacher is Yann-Gaël Guéhéneuc and the laboratory teacher is Fábio Petrillo. \\

We are three French exchange students; Fabien Berquez coming from \href{http://www.supelec.fr/}{Supélec} and Adrien Logut and Chathura Namalgamuwa coming from \href{http://ensimag.grenoble-inp.fr/}{ENSIMAG}. Fabien is a master's student in computer engineering, research-based, whereas Adrien and Chathura are master's student in computer engineering, professional-based.

\chapter{Presentation}

You can find all complete assignments (in French) in the \textit{assignment} folder. The objective of this project is to create a multi-platform web music player (eg. Soundcloud, Spotify, etc.) in Java. 

\section{Laboratory n$\degree$1}

The first laboratory's objective is to create a basis for the whole project. A basic music player has to be implemented, using at least 3 web services. \\

We started with implementing a service for Soundcloud because it was the easiest. Indeed, Soundcloud Web API can be used without authentication. It allowed us to test the code really quickly just using a \textit{client\_id}. 

%\section{Laboratory n$\degree$2}
%\section{Laboratory n$\degree$3}

\chapter{Sequence Diagram}

\chapter{Class Diagram}

\chapter{Implementation}

\chapter{Limitations}

\end{document}